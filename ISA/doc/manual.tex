\documentclass{article}

\usepackage[utf8]{inputenc}
\usepackage[a4paper, total={16cm, 24cm}]{geometry}
\usepackage[czech, slovak]{babel}
\usepackage{graphicx}
\usepackage[unicode]{hyperref}
\usepackage{float}


\renewcommand{\tt}[1]{\texttt{#1}}

\begin{document}


\begin{titlepage}
    \begin{center}
    
    \includegraphics[width=0.6\textwidth]{fit_logo.eps}
    
    \vspace{6cm}
    
    \huge{\textbf{ISA\,--\,Síťové aplikace a správa sítí}}
    \vspace{0.5cm}
    
    \Large{\textbf{DOKUMENTÁCIA K PROJEKTU}}
    \vspace{0.5cm}
    
    \Large{\textbf{Přenos souboru skrz skrytý kanál}}
    \vspace{2cm}
    
    \large{\textbf{Eduard Frlička \\ \texttt{\normalsize{\href{mailto:xfrlic00@stud.fit.vutbr.cz}{xfrlic00@stud.fit.vutbr.cz}}}}}
    \vspace{0.5cm}
    
    \vspace{1cm}
    
    
    \end{center}
    
    \end{titlepage}


\section{Vlastný prenosový protokol}
Pre správne fungovanie prenosu skrze ICMP pakety je potreba presne určiť strukturu prenosu.
Každý packet má echo identifikátor nastavený na hodnotu \tt{ECHO\_IDENTIFICATOR 0xEF00}.
Tým sa odlisuje tento prenosový protokol od ostatných ICMP packetov,
Hodnota \tt{code} je nastavovaná na jednu z hodnot enumu.

\begin{verbatim}
enum{
    initialPacket = 0x2,
    #define CODE_INITIAL initialPacket

    dataPacket = 0x4,
    #define CODE_DATA dataPacket
    
    terminalPacket = 0x8
    #define CODE_TERMINAL terminalPacket
};
\end{verbatim}

Prvý packet má \tt{code = CODE\_INITIAL}. Server pri prijatí tohto packetu otvorí súbor, alokuje buffer a uloží si štruktúru:

\begin{verbatim}
struct transmission {
    uint64_t hash;             /*Hash: primary key*/
    uint16_t nextSeq;          /*expected sequence number*/
    uint16_t offset_buff;      /*number of chars already stored in buffer*/
    FILE *file;                /*Pointer to file for writing*/
    u_char buff[BUFSIZ];       /*Buffer for writing. Size of "BUFSIZ" from <stdio.h>*/
    AES_KEY KeyDec;            /*decryption key used by AES_decrypt function*/
    struct transmission *next; /*pointer to next transmission*/
};

\end{verbatim}

Táto štruktúra zároveň funguje ako člen zreťazeného zoznamu a teda je možné príjmať viacero súborov naraz.
Packety ktoré obsahujú data majú \tt{code = CODE\_DATA} a posledný packet (korý je vždy prázdny), má \tt{code = CODE\_TERMINAL}.

\subsection{Vlastná hlavička}
Za ICMP hlavičkou sa v každom odoslanom packete nachádza hlavička definovaná ako:

\begin{verbatim}
struct SecretTransmisionProtocolHeader {
    uint64_t fileHash; /** hash calculated from time and filename */
    uint16_t datalen;  /** number of bytes before encryption */
} __attribute__((packed));
\end{verbatim}

\tt{fileHash} slúži na rozlíšenie prenosov v prípade príjmania viacerých súborov. 
Je to hash počítaný z názvu súboru a aktuálneho času.
V prípade kolízie hashu (čo je pri 64 bitovom čísle dosť nepravdepodobné), sa bude prenos započatý neskôr ignorovať.
\tt{datalen} je počet bytov odoslaných v packete, pred šifrovaním (keďze šifrovanie vyžaduje zarovnanie na 16B).

\section{Implementacia}

\subsection{Argumenty}
Program najprv parsuje argumenty z konzole.
Postuone sa prechadza argv pole a kontroluje sa vyskyt prepínačov \tt{-l, -s, -r}. 
Pri nájdení prepínača podľa potreby načíta hodnotu a pokračuje na ďalší prvok poľa.
Pokiaľ sa nenájde žiaden očakávaný prepínač, program sa ukončije s chybovou hláškou.
Na konci sa vyhodnotí ktoré prepínače sa nachádzali na príkazovom riadku, 
skontroluje sa ich platnosť (napr. či existuje súbor) a zavolá príslušná funkcia,
podľa toho, či sa našiel prepínač \tt{-l}.

\subsection{Client}
Modul client sa prevažne skladá z 2 funkcií: \tt{send\_file}, \tt{send\_packet}.

\subsubsection{\tt{send\_file}}
Funkcia \tt{send\_file} je volaná z mainu. 
Najprv získa potrebné informácie pre adresu, na ktorú sa bude súbor posielať.
Inicializuje sa socket a otvorí sa súbor.
Zistí sa dĺžka súboru (pre progress bar) a vypočíta sa hash (ktorý sa používa na rozlišovanie prenosov).
Pošle sa prvý packet, so sekvenčným číslom 0 a názvom súboru v datovej časti packetu.
Následne sa začne čítať súbor po 1024B blokoch.
Každý blok sa najprv zašifruje, a predá funkcii \tt{send\_packet}.
Po odoslaní všetkých packetov funkcia uvoľní alokovanú pamäť a zatvorí otvorený súbor.

\subsubsection{\tt{send\_packet}}
Funkcia alokuje dostatok pamäte pre data, icmp hlavičku a hlavičku protokolu.
Naplní icmp hlavičku a hlavičku protokolu potrebnými údajmi, prekopíruje dáta do packetu a vypočíta checksum.
Pomocou volania funkcie \tt{poll} počká až bude môcť odoslať packet na otvorený socket a následne odošle packet.
Po odoslaní čaká na prijmané packety.
Každý packet skontroluje a zistí či sa jedná o odpoveď na odoslaný packet. 
Kontroluje sa \tt{identifier} a \tt{sequence number}.
Až dôjde očakávaná odpoveď, uvoľní použivanú pamäť.

\subsection{Server}
Modul odchytáva a spracováva packety pomocou knižnice \tt{pcap}.
Sledujú sa všetky rozhrania a filter je nastavený na icmp echo request alebo icmp6 echo request.

\subsubsection{Spracovávanie packetov}
Keďze zachytávame packety z \tt{"any"} rohrania, prvá hlavička je SLL hlavička.
Zistí sa protokol (IPV4 / IPV6) a skontroluje sa checksum (pri IPV6 je potreba zostavit pseudoheader).
Pokiaľ je checksum správny, zavolá sa funkcia \tt{handleICMP}.

\subsubsection{\tt{handleICMP}}
Funkcia prečíta ICMP hlavičku, hlavičku protokolu a skontroluje či má identifikátor správnu hodnotu.
Skontroluje sekvenčné číslo pokiaľ sa jedná o už inicializovaný prenos.
Ak je hodnota sekvenčného čísla iná ako očakávana, sa packet buď preskočǐ alebo sa upraví hodnota očakávaného sekvenčného čísla.
V každom z týchto dvoch prípadov je na štandardný výstup vypísaná chybová hláška a program beží ďalej.
Následne sa na základe kódu z IMCP hlavičky zavolá jedna z funkcií \tt{handleDataPacket}, 
\tt{handleInitialPacket}, \tt{handleTerminalPacket}.
Funkcia \tt{handleTerminalPacket} sa stará o ukončenie prenosu.

\subsubsection{\tt{handleDataPacket}}
Funkcia dešifruje dáta z packetu, a následne ich zapíče do bufferu pre daný prenos.
Pokiaľ sa buffer naplní, vyprádzni sa volaním funkcie \tt{writeOut} a prípadné zvyšné dáta sa zapíšu do bufferu.

\subsubsection{\tt{handleInitialPacket}}
Funkcia alokuje a inicializuje štruktúru pre prenos, otvorí súbor a vypíše informačnú hlášku na štadnardný výstup.


\section{Rozsirenie}
Program podporuje zadanie viacerých súborov po prepínači \tt{-r}.

\end{document}